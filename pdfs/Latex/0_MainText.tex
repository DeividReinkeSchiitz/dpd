%!TEX root = 0_MainText.tex
%% MODELO DE LaTeX PARA TRABALHOS ACADÊMICOS
%% INSTRUÇÕES GERAIS:
%%    1. TODO O TEXTO NA FRENTE DO SIMBOLO '%' É COMENTÁRIO, ISTO É, ELE NÃO FAZ DIFERENÇA NO RESULTADO FINAL 
%%    2. NESTE MODELO, VOCÊS SÓ PRECISAM EDITAR DAS LINHAS 114 A 132 (INFORMAÇÕES DE CAPA) E DAS LINHAS 188 EM DIANTE (CORPO DO TRABALHO). O RESTO SÃO CONFIGURAÇÕES DE FORMATAÇÃO QUE PROVAVELMENTE NÃO SERÁ PRECISO MODIFICAR.
%%    3. MAIS INSTRUÇÕES DETALHADAS PODERÃO SER ENCONTRADAS NA PÁGINA profhelioh.wordpress.com. DÚVIDAS: heliohenrique@ufpr.br OU heliohenrique3@gmail.com

%% 4. Versão melhorada pelo prof. Luigi Galotto Jr - Dúvidas: luigi.galotto@ufms.br

% ------------------------------------------------------------------------
% ------------------------------------------------------------------------
% abnTeX2: Modelo de Relatório Técnico/Acadêmico em conformidade com 
% ABNT NBR 10719:2011 Informação e documentação - Relatório técnico e/ou
% científico - Apresentação
% ------------------------------------------------------------------------
% ------------------------------------------------------------------------

\documentclass[
	% -- opções da classe memoir --
	12pt,				% tamanho da fonte
	% openright,			% capítulos começam em pág ímpar (insere página vazia caso preciso)
        oneside,			% para impressão somente frente. Oposto a twoside (frente e verso)
	a4paper,			% tamanho do papel. 
	% -- opções da classe abntex2 --
	%chapter=TITLE,		% títulos de capítulos convertidos em letras maiúsculas
	%section=TITLE,		% títulos de seções convertidos em letras maiúsculas
	%subsection=TITLE,	% títulos de subseções convertidos em letras maiúsculas
	%subsubsection=TITLE,% títulos de subsubseções convertidos em letras maiúsculas
	% -- opções do pacote babel --
	english,			% idioma adicional para hifenização
	french,				% idioma adicional para hifenização
	spanish,			% idioma adicional para hifenização
	brazil,				% o último idioma é o principal do documento
	]{abntex2}

\usepackage{hyphenat}


% =======
% PACOTES
% =======

% ---
% Pacotes Fundamentais 
% ---
\usepackage{cmap}				% Mapear caracteres especiais no PDF
\usepackage{lmodern}			% Usa a fonte Latin Modern
\usepackage[T1]{fontenc}		% Selecao de codigos de fonte.
\usepackage[utf8]{inputenc}		% Codificacao do documento (conversão automática dos acentos)
\usepackage{indentfirst}		% Indenta o primeiro parágrafo de cada seção.
\usepackage{color}				% Controle das cores
\usepackage{graphicx}			% Inclusão de gráficos
\usepackage{siunitx}
\usepackage{listings}
\usepackage{color} %red, green, blue, yellow, cyan, magenta, black, white
\definecolor{mygreen}{RGB}{28,172,0} % color values Red, Green, Blue
\definecolor{mylilas}{RGB}{170,55,241}
\usepackage{upgreek}
\usepackage{gensymb}
\usepackage{tocloft}
\usepackage{amssymb}
\usepackage{stackengine} %letra sobre letra

\usepackage[savewrites,nopostdot,toc,acronym,symbols,nogroupskip]{glossaries}
\renewcommand{\glsnamefont}[1]{\textbf{#1}}
\setlength\LTleft{0pt}
\setlength\LTright{0pt}
\setlength\glsdescwidth{0.8\hsize}

\usepackage{enumitem}
\usepackage{dashbox}
\newcommand\dboxed[1]{\dbox{\ensuremath{#1}}}
%\usepackage{gensymb}           % The \degree command is provided by the gensymb package
% ---
%% gray box color
\usepackage{color}
\definecolor{lightgray}{gray}{0.75}

\newcommand\greybox[1]{%
  \vskip\baselineskip%
  \par\noindent\colorbox{lightgray}{%
    \begin{minipage}{\textwidth}#1\end{minipage}%
  }%
  \vskip\baselineskip%
}
%%
% ---
% Minhas modificações
% ---
\usepackage{float} % Imagens com H
%\interfootnotelinepenalty=10000
\usepackage{chngcntr} % Captions de acordo com as seções
\counterwithout{figure}{chapter} % Numeracao corrida
% \counterwithin{figure}{section} % Numeracao por capitulo. Ex.: Figura 3.2
\counterwithout{table}{chapter} % Numeracao corrida

\usepackage{mathrsfs,amsmath} % F da transformada de Fourier
\usepackage{tipa}
\usepackage{textcomp} % text symbols, like registered or trademark
\usepackage{subcaption} % subfigura

\usepackage{mathtools} % Utilizar matrix e também condicionais
%\usepackage[titles]{tocloft} % Espaçamento na lista de figuras
\cftsetindents{figure}{0em}{4em}
\cftsetindents{table}{0em}{4em}

% ---
% Pacotes para gráficos com pgfplots/tikz
% ---
\usepackage{pgfplots}
\pgfplotsset{compat=1.18}
\usepackage{tikz}

% ---
% Pasta de imagens
% ---
\graphicspath{{./Figuras/}}


% ---
% Opção de Lista Automática das abreviações
%\makeglossaries
%\renewcommand*\glspostdescription{\cftdotfill{\cftsecdotsep}}
%\newglossaryentry{latex}
{
    name=latex,
    description={Is a mark up language specially suited for scientific documents}
}


\newacronym{cc}{CC}{Corrente Contínua}
\newacronym{sqt}{SQT}{soma dos quadrados totais}
\newacronym{pid}{PID}{proporcional-integral-derivativo}
\newacronym{ols}{OLS}{\textit{ordinary least squares}}
\newacronym{sqe}{SQE}{soma dos quadrados explicados}
\newacronym{sqr}{SQR}{soma dos quadrados dos resíduos}
\newacronym{s}{$\displaystyle s$}{desvio padrão amostral}
\newacronym{pwm}{PWM}{pulse width modulation} %importa as abreviações

% ---
% Pacotes adicionais, usados no anexo do modelo de folha de identificação
% ---
\usepackage{multicol}
\usepackage{multirow}
% ---
% ---
% Pacotes adicionais, usados apenas no âmbito do Modelo Canônico do abnteX2
% ---
\usepackage{lipsum}				% para geração de dummy text
% ---

% ---
% Pacotes de citações
% ---
%\usepackage[brazilian,hyperpageref]{backref}  % Paginas com as citações nas Biblio

% Escolher um dos dois padrões a seguir:
%\usepackage[num]{abntex2cite}	              % Citações numérica padrão ABNT
\usepackage[alf]{abntex2cite}	              % Citações nominais padrão ABNT

% ---

% --- 
% CONFIGURAÇÕES DE PACOTES
% ---
% Configurações do pacote backref
% Usado sem a opção hyperpageref de backref
%\renewcommand{\backrefpagesname}{Citado na(s) página(s):~}
% Texto padrão antes do número das páginas
%\renewcommand{\backref}{}
% Define os textos da citação
%\renewcommand*{\backrefalt}[4]{
	%\ifcase #1 %
		%Nenhuma citação no texto.%
	%\or
		%Citado na página #2.%
	%\else
		%Citado #1 vezes nas páginas #2.%
	%\fi}%
% ---

% ---
% Informações de dados para CAPA e FOLHA DE ROSTO
% ---
\titulo{Uma Abordagem Matheurística baseada em Large Neighborhood Search para o Problema da Distribuição de Disciplinas}
\autor{Deivid Reinke Schutz}
\local{Campo Grande - MS}
\data{23 de Janeiro de 2026}
\orientador{Profa. Dr. Edna Ayako Hoshino}
\membrodabancaA{Prof. Dr. Membro da banca 1}
\membrodabancaB{Prof. Dr. Membro da banca 2}
\instituicao{%
    \mbox{UNIVERSIDADE FEDERAL DE MATO GROSSO DO SUL}
    \mbox{FACULDADE DE COMPUTAÇÃO}
    %\mbox{Dissertação de mestrado do curso de Engenharia Elétrica}
    \mbox{CURSO DE CIÊNCIA DA COMPUTAÇÃO}
    %\mbox{MESTRADO EM SISTEMAS DE ENERGIA}
}
\tipotrabalho{Dissertação}

% O preambulo deve conter o tipo do trabalho, o objetivo, 
% o nome da instituição e a área de concentração 
\preambulo{
Trabalho de Conclusão de Curso apresentado como exigência para obtenção do grau de Bacharelado em Ciência da Computação da Universidade Federal de Mato Grosso do Sul – UFMS.}
% ---

% ---
% Configurações de aparência do PDF final
% alterando o aspecto da cor azul
\definecolor{blue}{RGB}{41,5,195}

% informações do PDF
\makeatletter
\hypersetup{
     	%pagebackref=true,
		pdftitle={\@title}, 
		pdfauthor={\@author},
    	pdfsubject={\imprimirpreambulo},
	    pdfcreator={LaTeX with abnTeX2},
		pdfkeywords={abnt}{latex}{abntex}{abntex2}{Dissertação de Mestrado}, 
		colorlinks=true,       		% false: boxed links; true: colored links
    	linkcolor=blue,          	% color of internal links
    	citecolor=blue,        		% color of links to bibliography
    	filecolor=magenta,      	% color of file links
		urlcolor=blue,
		bookmarksdepth=4
}
\makeatother
% Fallbacks for bibliography produced commands that may use htmladdnormallink
% Some .bbl files (abntex styles) use \htmladdnormallink; if undefined, map to \href
\providecommand{\htmladdnormallink}[2]{\href{#1}{#2}}
\providecommand{\htmladdnormallinkfoot}[2]{\href{#1}{#2}}
\providecommand{\htmladdnormallinknew}[2]{\href{#1}{#2}}
% --- 

% --- 
% Espaçamentos entre linhas e parágrafos 
% --- 
% O tamanho do parágrafo é dado por:
\setlength{\parindent}{1.3cm}

% Controle do espaçamento entre um parágrafo e outro:
\setlength{\parskip}{0.2cm}  % tente também \onelineskip

% ---
% Compila o indice
% ---
\makeindex
% ---
\usepackage{pdfpages}
% ----
% Início do documento
% ----
\begin{document}

% Retira espaço extra obsoleto entre as frases.
\frenchspacing 

% ----------------------------------------------------------
% ELEMENTOS PRÉ-TEXTUAIS
% ----------------------------------------------------------
% \pretextual

% ---
% Capa
% ---
\imprimircapa
% ---

% ---
% Folha de rosto
\imprimirfolhaderosto
%\imprimirfolhaderosto* % (use esta opção com * quando houver ficha catalográfica)

% Folha de aprovação
\imprimirfolhadeaprovacao
% ---


% ---
% Agradecimentos
% ---
\begin{agradecimentos} % (opcional)
% Adicionar o texto
   
\end{agradecimentos}
% ---

% ---
% RESUMO
% ---
% resumo na língua vernácula (obrigatório)
\begin{resumo} %% AQUI COMEÇA A PÁGINA DE RESUMO
	Este trabalho aborda o Problema da Distribuição de Disciplinas (DPD), que consiste em atribuir disciplinas a professores respeitando restrições de alocação e limites de carga horária, visando maximizar a qualidade da atribuição. O problema é modelado como um programa linear inteiro e resolvido no SCIP via Branch-and-Bound.

	Como contribuição, propõe-se e avalia-se uma matheurística baseada em Large Neighborhood Search (LNS) no esquema fix-and-relax, que explora uma solução incumbente e define uma vizinhança ao fixar a maior parte das variáveis binárias $x_{p,c}$, liberando uma fração controlada para reotimização. A taxa de destruição (\texttt{lns\_perc}), a ordenação de candidatos (\texttt{avgPreferenceWeight}) e o tempo máximo do sub-MIP são analisados quanto ao impacto na qualidade da solução e nas métricas de desempenho observadas: tempo, gap e nós remanescentes (nodes-left).

\noindent
	{\sloppy \textbf{Palavras-chave}: Distribuição de Disciplinas, Programação Linear Inteira, Branch-and-Bound, Matheurística, Large Neighborhood Search, Fix-and-Relax, SCIP.\par}\fussy
\end{resumo} %AQUI TERMINA A PÁGINA DE RESUMO
% ---

\cleardoublepage %% Pula página
% ---
% inserir lista de ilustrações
% ---
\listoffigures* %% o * indica que não será incluso no sumário
\cleardoublepage %% Pula página
% ---

% ---
% inserir lista de tabelas
% ---
\listoftables*
\cleardoublepage


% --- 
% inserir lista automática de abreviaturas e siglas
% ---
%\printglossary%[title=Abreviações,nonumberlist,type=acronym,style=long]
%\cleardoublepage
% ---

% ---
% inserir lista de abreviaturas e siglas
% ---
%\begin{siglas} % OPCIONAL
%    \item [HEV] Veículo Elétrico Híbrido    
%\end{siglas}
% ---

% ---
% inserir lista de símbolos
% ---
%\begin{simbolos} % OPCIONAL
%    \item[$ \longleftrightarrow $] Conectivo bicondicional "Se e somente se"
%    \item[$ f(t) $] Função no domínio do tempo
%    \item[$ F(S) $] Função no domínio da frequência
%    \item[$ \mathscr{F} $] Transformada direta de Fourier
%    \item[$ \mathscr{F}^{-1} $] Transformada inversa de Fourier
%\end{simbolos}
% ---

% ---
% inserir o sumario
% ---
\tableofcontents*
% ---

% ----------------------------------------------------------
% ELEMENTOS TEXTUAIS  (necessário para incluir número nas páginas)
% ----------------------------------------------------------
\textual

% ----------------------------------------------------------
% Introdução
% ----------------------------------------------------------
\chapter{Introdução}

\section{Contextualização}

O planejamento acadêmico semestral envolve diversas decisões interdependentes, entre as quais se destaca a atribuição de disciplinas a professores. Em cursos de graduação, essa distribuição precisa respeitar restrições institucionais (por exemplo, alocação única de cada disciplina, limites de carga horária por período e por ano e regras internas de oferta), ao mesmo tempo em que busca atender critérios de qualidade associados à afinidade do docente com o conteúdo, histórico de atuação e preferências.

Neste trabalho considera-se o 
Problema da Distribuição de Disciplinas (DPD), no qual o objetivo é construir uma atribuição disciplina--professor que maximize a qualidade da alocação sob restrições operacionais. O DPD pode ser modelado como um problema de otimização combinatória e, em particular, como um programa linear inteiro (PLI) com variáveis binárias indicando se um professor $p$ é designado para uma disciplina $c$. Modelos desse tipo podem ser resolvidos por métodos exatos, como Branch-and-Bound, disponíveis em solvers modernos como o SCIP.

Apesar de a modelagem via PLI permitir incorporar restrições e critérios de forma sistemática, instâncias de maior porte e/ou com restrições mais detalhadas podem tornar a obtenção de boas soluções em tempo limitado um desafio prático. Isso motiva o uso de estratégias que combinem a força dos métodos exatos com mecanismos heurísticos de intensificação.

\section{Motivação e Justificativa}

Na prática, a distribuição de disciplinas é um processo recorrente e sensível: pequenas mudanças em regras, em ofertas ou no quadro docente podem alterar significativamente o espaço de soluções. Além disso, a necessidade de respostas em prazos curtos torna desejável obter soluções viáveis de boa qualidade rapidamente, mesmo quando a prova de otimalidade não é alcançada.

Uma linha de pesquisa que atende a essa necessidade é o uso de matheurísticas, isto é, heurísticas baseadas em programação matemática, que exploram subproblemas de PLI (ou relaxações) como operadores dentro de um procedimento de busca. Em particular, a Large Neighborhood Search (LNS) é uma matheurística de destruição e reparação: parte-se de uma solução incumbente, 
``destrói-se'' uma parte controlada das decisões e, em seguida, reotimiza-se o subproblema resultante para ``reparar'' a solução. Esse paradigma tem sido utilizado com sucesso em diferentes problemas combinatórios e também em contextos que integram técnicas exatas e heurísticas \cite{neto2023matheuristicas}.

No contexto do DPD, a LNS é especialmente atrativa por permitir explorar vizinhanças grandes preservando a maior parte da estrutura da incumbente. Nesta monografia, implementa-se uma LNS no estilo 
fix-and-relax no SCIP: fixa-se a maior parte das variáveis binárias conforme a incumbente e libera-se uma fração das atribuições para reotimização, gerando um subproblema inteiro (sub-MIP) resolvido com limite de tempo. A estratégia é integrada ao processo de Branch-and-Bound, buscando intensificação por meio de melhorias sucessivas da incumbente.

Além da análise interna da LNS, este trabalho também considera uma comparação com uma heurística GRASP desenvolvida especificamente para o DPD, bem como uma estratégia híbrida na qual a LNS é aplicada como etapa de melhoria a partir de soluções construídas pelo GRASP (GRASP|LNS). Essa comparação visa evidenciar, nas métricas de tempo, gap e nós remanescentes, em quais cenários a intensificação via LNS agrega valor em relação a uma abordagem construtiva/melhorativa dedicada \cite{jhonatan_grasp_dpd}.

\section{Objetivos}

\subsection{Objetivo Geral}

Propor, implementar e avaliar uma matheurística baseada em LNS, integrada ao solver SCIP, para obter soluções de boa qualidade para o DPD em tempo limitado.

\subsection{Objetivos Específicos}

\begin{itemize}
    \item modelar o DPD como um PLI e integrá-lo ao framework de solução do SCIP, utilizando Branch-and-Bound como abordagem exata de referência;
    \item implementar uma matheurística LNS do tipo fix-and-relax, baseada em destruição parcial da incumbente e reparação via resolução de um sub-MIP em uma instância secundária (sub-SCIP);
    \item analisar o impacto de parâmetros internos da LNS, incluindo a taxa de destruição, o sentido de ordenação dos candidatos e o tempo máximo do subproblema;
    \item avaliar o desempenho por meio das métricas observadas nos experimentos: tempo, gap e nós remanescentes (nodes-left);
    \item comparar a estratégia LNS com abordagens alternativas consideradas no trabalho (relaxação e a heurística GRASP), incluindo a variação híbrida GRASP|LNS, discutindo cenários em que a LNS é mais efetiva.
\end{itemize}

\section{Organização do Trabalho}

O restante deste trabalho está organizado da seguinte forma. No Capítulo~2 é apresentada a fundamentação teórica, cobrindo conceitos de otimização inteira e heurísticas/matheurísticas relevantes ao estudo. O Capítulo~3 define o DPD e descreve o modelo matemático exato adotado. O Capítulo~4 apresenta os resultados experimentais e a análise computacional, incluindo a avaliação das configurações internas da LNS e comparações de desempenho. Por fim, o Capítulo~5 conclui o trabalho e discute limitações e possibilidades de trabalhos futuros.


% ----------------------------------------------------------
% Embasamento Teórico
% ----------------------------------------------------------
\chapter{Fundamentação Teórica}

\section{Otimização Combinatória e Programação Linear Inteira (PLI)}

% Para citações que fazem parte da sentença, conforme a ABNT, utilizar o comando \citeonline{}

Problemas de alocação em ambientes acadêmicos podem ser formulados como problemas de otimização combinatória, nos quais se busca selecionar uma atribuição que maximize uma medida de qualidade (por exemplo, preferências) e, ao mesmo tempo, respeite restrições operacionais (carga horária, conflitos e requisitos mínimos). Na literatura, problemas próximos incluem o \textit{class--faculty assignment} e variantes de alocação docente com preferências \cite{al_yakoob_sherali_2006,al_yakoob_sherali_2013,domenech_lusa_2016}, bem como estudos em \textit{course timetabling} \cite{carter_laporte_1998,avella_vasilyev_2005}.

Uma forma padrão de modelar tais problemas é por Programação Linear Inteira (PLI), em que variáveis de decisão (frequentemente binárias) representam escolhas discretas, e a função objetivo quantifica a qualidade da solução. De modo geral, um modelo pode ser escrito como
\[
\max\ \sum_{p \in P}\sum_{c \in C} w_{p,c}\,x_{p,c}\quad \text{sujeito a restrições lineares e } x_{p,c}\in\{0,1\},
\]
onde $x_{p,c}=1$ indica que o professor $p$ é designado à disciplina $c$, e $w_{p,c}$ representa o peso/preferência. A PLI permite representar restrições de compatibilidade e limites de capacidade de forma explícita, mantendo uma interpretação direta do modelo \cite{vanderbei_2014,wolsey_2020}.

\section{O Método Branch-and-Bound e Relaxação Linear}

Modelos de PLI são, em geral, NP-difíceis, de modo que solucionadores utilizam métodos exatos baseados em limites e enumeração inteligente. Um componente central é a relaxação linear, obtida ao substituir a integralidade por $0\le x\le 1$, produzindo um problema de Programação Linear (PL) que fornece um limite para a solução inteira ótima \cite{vanderbei_2014,wolsey_2020}.

O Branch-and-Bound (B\&B) explora uma árvore de subproblemas: a cada nó, resolve-se uma relaxação para obter um limite e, quando necessário, ramifica-se em subproblemas impondo decisões de integralidade (por exemplo, $x_{p,c}=0$ ou $x_{p,c}=1$). Nós cujos limites não superam a melhor solução inteira conhecida podem ser podados, reduzindo a busca \cite{wolsey_2020}. Na prática, solucionadores modernos integram B\&B com técnicas adicionais; neste trabalho, a resolução exata é realizada com o SCIP \cite{gamrath_2020_scip}.

\section{Heurísticas e Metaheurísticas}

Embora métodos exatos forneçam garantias de otimalidade, instâncias reais podem apresentar grande dimensão e múltiplas restrições, tornando necessário recorrer a estratégias aproximadas. Heurísticas constroem boas soluções em tempo reduzido, enquanto metaheurísticas organizam mecanismos de exploração e intensificação para escapar de ótimos locais e melhorar soluções iterativamente.

Para problemas de alocação docente, há abordagens heurísticas e metaheurísticas que exploram diferentes representações e movimentos de vizinhança, como algoritmos genéticos e buscas locais especializadas \cite{gunawan_ng_ong_2008,hosny_2012,szwarc_2018}. Esse corpo de técnicas é especialmente útil quando se deseja produzir soluções de boa qualidade sob limites de tempo, ou quando o modelo exato é utilizado como referência/validação.

\subsection{Greedy Randomized Adaptive Search Procedure (GRASP)}

O GRASP é uma metaheurística em duas fases: (i) uma construção gulosa aleatorizada, que produz uma solução inicial por escolhas guiadas por um critério de qualidade com componente probabilística; e (ii) uma busca local, que tenta melhorar a solução por movimentos em uma vizinhança definida. A aleatoriedade permite amostrar diferentes regiões do espaço de soluções, enquanto a busca local promove intensificação.

No contexto do Problema da Distribuição de Disciplinas (DPD), uma estratégia GRASP é um caminho natural por combinar regras gulosas (por exemplo, priorizar pares professor--disciplina com maior peso) com fases de melhoria que reatribuem disciplinas para reduzir violações e aumentar a qualidade \cite{jhonatan_grasp_dpd}.

\section{Matheurísticas}

Matheurísticas combinam programação matemática com componentes heurísticos/metaheurísticos, buscando aproveitar simultaneamente a força de modelagem e os limites fornecidos por modelos exatos, e a flexibilidade de estratégias de busca em grandes espaços combinatórios \cite{ball_2011,boschetti_letchford_maniezzo_2023,maniezzo_stutzle_voss_2021}. Em geral, a ideia é usar submodelos (ou modelos restritos) como operadores de melhoria, resolver subproblemas por MIP/CPLEX/SCIP, ou ainda usar informações do solucionador (limites, incumbentes, cortes) para orientar a busca.

Em problemas do tipo mochila e variantes, por exemplo, há linhas de trabalho que empregam resoluções exatas em subestruturas, gerando melhorias sistemáticas em torno de incumbentes \cite{dambrosio_laureana_raiconi_vitale_2023,jovanovic_voss_2024}. Essa mesma filosofia se estende ao DPD: parte-se de um modelo de PLI e explora-se uma vizinhança definida sobre variáveis binárias, reotimizando subproblemas com tempo controlado.

\subsection{Large Neighborhood Search (LNS) e Fix-and-Relax}

O Large Neighborhood Search (LNS) alterna etapas de \textit{destroy} e \textit{repair}: uma porção da solução corrente é liberada (ou removida) e, em seguida, o problema restrito é resolvido para reparar e potencialmente melhorar a solução. A principal vantagem é permitir movimentos de grande alcance, mantendo a viabilidade por meio de um procedimento de reparo eficaz \cite{ahuja_ergun_orlin_punnen_2002,pisinger_ropke_2010}. Uma implementação prática e amplamente adotada é usar um solucionador MIP na etapa de reparo, o que transforma o LNS em uma matheurística.

O esquema \textit{fix-and-relax} pode ser visto como uma forma estruturada de definir vizinhanças: fixa-se a maior parte das variáveis (mantendo decisões da incumbente) e relaxa-se um subconjunto selecionado para reotimização, normalmente sob um limite de tempo. Esse desenho permite controlar o tamanho da vizinhança e o custo computacional de cada iteração, ao mesmo tempo em que preserva a capacidade do MIP de recombinar decisões de modo consistente. Assim, o método proposto neste trabalho utiliza o SCIP \cite{gamrath_2020_scip} como motor de reotimização em vizinhanças geradas a partir de uma solução incumbente.

Por fim, este TCC se insere na linha de abordagens para o DPD que exploram tanto modelos exatos quanto estratégias de melhoria, dialogando com trabalhos recentes no problema \cite{silva2024abordagem_exata_dpd} e com a motivação geral de matheurísticas de melhoria para problemas combinatórios de grande porte \cite{silva2024matheuristica_mochila}.

% ----------------------------------------------------------
% Desenvolvimento
% ----------------------------------------------------------

\input{3_Metodologia.tex}

\chapter{Resultados e Análise Computacional} \label{ch:resultados}

\section{Instâncias e Ambiente de Testes}

Os experimentos computacionais foram conduzidos utilizando um conjunto de 42 instâncias de teste geradas sinteticamente, abrangendo diferentes cenários do Problema da Distribuição de Disciplinas (DPD). As instâncias variam em tamanho (62 a 214 disciplinas), número de professores (22 a 69), complexidade das restrições de carga horária e distribuição de preferências entre áreas de conhecimento.

Para cada experimento, foram selecionadas aleatoriamente 10 instâncias dentre as 42 disponíveis, garantindo representatividade estatística e mantendo o tempo total de experimentação viável. As mesmas instâncias foram utilizadas em todas as configurações testadas de um mesmo parâmetro, assegurando comparabilidade direta dos resultados.

Todas as execuções foram realizadas com limite de tempo global de 400 segundos por instância. O solver utilizado foi o SCIP 7.0 \cite{gamrath_2020_scip}, integrado com as heurísticas LNS e GRASP desenvolvidas neste trabalho. A calibração dos parâmetros da heurística LNS foi conduzida de forma incremental, conforme a abordagem \emph{one-factor-at-a-time}: primeiro determinou-se a melhor taxa de destruição, em seguida o melhor sentido de ordenação e, finalmente, o melhor limite de tempo para o sub-MIP. Essa estratégia permite isolar o efeito de cada parâmetro e identificar a configuração mais adequada de forma sistemática.

Para a seleção do melhor valor de cada parâmetro, adotou-se a seguinte hierarquia de critérios de desempate: (i)~tempo de execução, (ii)~gap de otimalidade e (iii)~número de nós restantes na árvore de Branch-and-Bound (\emph{nodes left}). Quando os critérios de maior prioridade apresentaram empate entre as configurações, o critério subsequente foi utilizado como desempate.

\section{Análise Interna das Configurações da LNS}

\subsection{Impacto da Taxa de Destruição (10\%, 25\%, 50\% e 75\%)}

A primeira análise experimental investigou o impacto da taxa de destruição (\texttt{lns\_perc}) na qualidade das soluções e no desempenho computacional. Foram testadas quatro configurações: 10\%, 25\%, 50\% e 75\% de destruição, mantendo fixos o tempo máximo do LNS (30s por chamada), a ordenação (crescente) e o tempo total de execução (400s).

% ============================================
% TABELAS - RESULTADOS DOS EXPERIMENTOS LNS
% ============================================

% Pacotes necessários: booktabs, multirow, array

% ============================================
% TABELA 1: Impacto da Taxa de Destruição - Tempo
% ============================================
\begin{table}[htbp]
\centering
\caption{Tempo de execução (segundos) para diferentes taxas de destruição do LNS. Configuração: ordenação crescente, tempo LNS = 30s, limite global = 400s.}
\label{tab:lns_cand_tempo}
\small
\begin{tabular}{@{}lrrrrrrrr@{}}
\toprule
\textbf{Instância} & \textbf{Disciplinas} & \textbf{Profs} & \textbf{Relax.} & \textbf{10\%} & \textbf{25\%} & \textbf{50\%} & \textbf{75\%} & \textbf{Melhor} \\
\midrule
input2  & 79  & 24 & 4,04    & 17,07  & 17,66  & 16,98  & 17,04  & Relax. \\
input5  & 84  & 29 & 113,03  & 145,27 & 145,04 & 145,71 & 143,63 & Relax. \\
input6  & 113 & 31 & 1,07    & 2,47   & 2,40   & 2,33   & 2,22   & Relax. \\
input7  & 99  & 28 & 400,02  & 400,02 & 400,02 & 400,02 & 400,02 & Empate \\
input9  & 133 & 41 & 17,06   & 72,52  & 72,33  & 72,57  & 73,48  & Relax. \\
input13 & 125 & 39 & 29,13   & 60,21  & 59,56  & 59,04  & 59,43  & Relax. \\
input19 & 114 & 39 & 2,11    & 1,39   & 1,44   & 1,42   & 1,47   & LNS 10\% \\
input20 & 113 & 40 & 0,95    & 4,69   & 4,56   & 4,59   & 5,02   & Relax. \\
input26 & 185 & 60 & 33,73   & 25,50  & 24,67  & 24,81  & 24,38  & LNS 75\% \\
input34 & 137 & 42 & 21,92   & 26,68  & 26,08  & 26,38  & 26,41  & Relax. \\
\midrule
\textbf{Média}  & 114 & 39 & 19,49 & 25,38 & 25,38 & 25,59 & 25,39 & Relax. \\
\textbf{Total}  & 1.296 & 412 & \textbf{623,05} & 755,82 & 753,76 & 753,84 & 753,10 & \textbf{Relax.} \\
\bottomrule
\end{tabular}
\end{table}

% ============================================
% TABELA 2: Impacto da Taxa de Destruição - Nodes Left
% ============================================
\begin{table}[htbp]
\centering
\caption{Número de nós restantes na árvore de Branch-and-Bound para diferentes taxas de destruição. Valores menores indicam maior progresso em direção à otimalidade.}
\label{tab:lns_cand_nodes}
\small
\begin{tabular}{@{}lrrrrrrrr@{}}
\toprule
\textbf{Instância} & \textbf{Disciplinas} & \textbf{Profs} & \textbf{Relax.} & \textbf{10\%} & \textbf{25\%} & \textbf{50\%} & \textbf{75\%} & \textbf{Melhor} \\
\midrule
input2  & 79  & 24 & 0      & 0      & 0      & 0      & 0      & -- \\
input5  & 84  & 29 & 0      & 0      & 0      & 0      & 0      & -- \\
input6  & 113 & 31 & 0      & 0      & 0      & 0      & 0      & -- \\
input7  & 99  & 28 & 66.230 & 62.628 & 60.540 & 61.056 & \textbf{58.921} & \textbf{75\%} \\
input9  & 133 & 41 & 0      & 0      & 0      & 0      & 0      & -- \\
input13 & 125 & 39 & 0      & 0      & 0      & 0      & 0      & -- \\
input19 & 114 & 39 & 0      & 0      & 0      & 0      & 0      & -- \\
input20 & 113 & 40 & 0      & 0      & 0      & 0      & 0      & -- \\
input26 & 185 & 60 & 0      & 0      & 0      & 0      & 0      & -- \\
input34 & 137 & 42 & 0      & 0      & 0      & 0      & 0      & -- \\
\midrule
\textbf{Média}  & 114 & 39 & 6.623 & 6.263 & 6.054 & 6.106 & 5.892 & LNS 75\% \\
\textbf{Total}  & 1.296 & 412 & 66.230 & 62.628 & 60.540 & 61.056 & \textbf{58.921} & \textbf{LNS 75\%} \\
\bottomrule
\end{tabular}
\end{table}

% ============================================
% TABELA 3: Impacto da Ordenação - Tempo
% ============================================
\begin{table}[htbp]
\centering
\caption{Tempo de execução (segundos) comparando ordenação crescente e decrescente. Configuração: 75\% de destruição, tempo LNS = 30s, limite global = 400s.}
\label{tab:lns_order_tempo}
\small
\begin{tabular}{@{}lrrrrrr@{}}
\toprule
\textbf{Instância} & \textbf{Disciplinas} & \textbf{Profs} & \textbf{Relax.} & \textbf{Crescente} & \textbf{Decrescente} & \textbf{Melhor} \\
\midrule
input4  & 123 & 46 & 400,05 & 400,04 & 400,05 & Crescente \\
input5  & 214 & 69 & 114,18 & 140,64 & 141,47 & Crescente \\
input12 & 115 & 37 & 5,79   & 25,70  & 25,84  & Crescente \\
input15 & 94  & 30 & 400,03 & 400,02 & 400,02 & Decrescente \\
input16 & 89  & 28 & 3,07   & 0,72   & 0,72   & Decrescente \\
input17 & 108 & 38 & 400,04 & 400,04 & 400,03 & Decrescente \\
input20 & 104 & 37 & 0,90   & 4,40   & 4,32   & Decrescente \\
input33 & 104 & 36 & 400,03 & 400,03 & 400,03 & Decrescente \\
input37 & 152 & 53 & 68,00  & 31,61  & 31,62  & Crescente \\
input38 & 62  & 22 & 0,16   & 0,28   & 0,28   & Decrescente \\
\midrule
\textbf{Média}  & 106 & 37 & 91,09 & 180,31 & 180,38 & Crescente \\
\textbf{Total}  & 1.271 & 433 & \textbf{1.792,24} & 1.803,48 & 1.804,48 & \textbf{Relax.} \\
\bottomrule
\end{tabular}
\end{table}

% ============================================
% TABELA 4: Impacto da Ordenação - Nodes Left
% ============================================
\begin{table}[htbp]
\centering
\caption{Número de nós restantes comparando ordenação crescente e decrescente.}
\label{tab:lns_order_nodes}
\small
\begin{tabular}{@{}lrrrrrr@{}}
\toprule
\textbf{Instância} & \textbf{Disciplinas} & \textbf{Profs} & \textbf{Relax.} & \textbf{Crescente} & \textbf{Decrescente} & \textbf{Melhor} \\
\midrule
input4  & 123 & 46 & 19.797 & 19.541 & 19.507 & Decrescente \\
input5  & 214 & 69 & 0      & 0      & 0      & -- \\
input12 & 115 & 37 & 0      & 0      & 0      & -- \\
input15 & 94  & 30 & 64.711 & \textbf{61.546} & 62.452 & \textbf{Crescente} \\
input16 & 89  & 28 & 0      & 0      & 0      & -- \\
input17 & 108 & 38 & 4.673  & 4.667  & 4.652  & Decrescente \\
input20 & 104 & 37 & 0      & 0      & 0      & -- \\
input33 & 104 & 36 & 50.279 & \textbf{46.243} & 47.063 & \textbf{Crescente} \\
input37 & 152 & 53 & 0      & 0      & 0      & -- \\
input38 & 62  & 22 & 0      & 0      & 0      & -- \\
\midrule
\textbf{Média}  & 106 & 37 & 13.946 & 13.200 & 13.367 & Crescente \\
\textbf{Total}  & 1.271 & 433 & 139.460 & \textbf{131.997} & 133.674 & \textbf{Crescente} \\
\bottomrule
\end{tabular}
\end{table}

% ============================================
% TABELA 5: Impacto do Tempo Máximo LNS - Tempo
% ============================================
\begin{table}[htbp]
\centering
\caption{Tempo de execução (segundos) para diferentes limites de tempo do sub-MIP. Configuração: 75\% de destruição, ordenação crescente, limite global = 400s.}
\label{tab:lns_time_tempo}
\small
\begin{tabular}{@{}lrrrrrrrrr@{}}
\toprule
\textbf{Inst.} & \textbf{Disc.} & \textbf{Prof.} & \textbf{Relax.} & \textbf{10s} & \textbf{50s} & \textbf{100s} & \textbf{200s} & \textbf{Melhor} \\
\midrule
input6  & 79  & 24 & 1,19   & 2,40   & 2,14   & 2,15   & \textbf{2,14}   & \textbf{200s} \\
input14 & 99  & 28 & 14,21  & 22,31  & 68,47  & 118,44 & 168,12 & Relax. \\
input15 & 94  & 30 & 400,03 & 400,02 & 400,02 & \textbf{76,31} & 77,09 & \textbf{100s} \\
input21 & 104 & 31 & 6,98   & \textbf{3,02}   & 3,04   & 3,05   & 3,10   & \textbf{10s} \\
input22 & 133 & 41 & 400,05 & 400,04 & 400,04 & 400,04 & 400,04 & 10s \\
input23 & 118 & 35 & 1,55   & \textbf{11,41} & 12,80  & 12,66  & 12,50  & \textbf{10s} \\
input27 & 125 & 39 & 4,93   & 7,45   & 7,34   & 7,33   & \textbf{7,31}   & \textbf{200s} \\
input29 & 103 & 39 & 400,03 & 400,03 & 19,42  & 19,40  & \textbf{19,39}  & \textbf{200s} \\
input33 & 104 & 36 & 400,03 & 400,03 & 400,03 & 400,03 & 400,03 & 200s \\
input38 & 62  & 22 & 0,16   & 0,28   & 0,28   & \textbf{0,28}   & 0,28   & \textbf{100s} \\
\midrule
\textbf{Média} & 104 & 33 & 162,92 & 164,77 & 162,11 & 160,27 & \textbf{159,46} & \textbf{200s} \\
\textbf{Total} & 1.125 & 358 & 1.629,16 & 1.647,01 & 1.313,60 & \textbf{1.039,68} & 1.090,00 & \textbf{100s} \\
\bottomrule
\end{tabular}
\end{table}

% ============================================
% TABELA 6: Impacto do Tempo Máximo LNS - Nodes Left
% ============================================
\begin{table}[htbp]
\centering
\caption{Número de nós restantes para diferentes limites de tempo do sub-MIP.}
\label{tab:lns_time_nodes}
\small
\begin{tabular}{@{}lrrrrrrrrr@{}}
\toprule
\textbf{Inst.} & \textbf{Disc.} & \textbf{Prof.} & \textbf{Relax.} & \textbf{10s} & \textbf{50s} & \textbf{100s} & \textbf{200s} & \textbf{Melhor} \\
\midrule
input6  & 79  & 24 & 0      & 0      & 0      & 0      & 0      & -- \\
input14 & 99  & 28 & 0      & 0      & 0      & 0      & 0      & -- \\
input15 & 94  & 30 & 54.355 & 63.056 & 59.819 & \textbf{0} & \textbf{0} & \textbf{100s} \\
input21 & 104 & 31 & 0      & 0      & 0      & 0      & 0      & -- \\
input22 & 133 & 41 & 21.528 & 29.161 & \textbf{25.100} & 50.411 & 33.000 & \textbf{50s} \\
input23 & 118 & 35 & 0      & 0      & 0      & 0      & 0      & -- \\
input27 & 125 & 39 & 0      & 0      & 0      & 0      & 0      & -- \\
input29 & 103 & 39 & 12.643 & 4.347  & \textbf{0} & \textbf{0} & \textbf{0} & \textbf{50s} \\
input33 & 104 & 36 & 47.460 & 48.110 & 45.505 & 1.524  & \textbf{539} & \textbf{200s} \\
input38 & 62  & 22 & 0      & 0      & 0      & 0      & 0      & -- \\
\midrule
\textbf{Média} & 104 & 33 & 13.599 & 14.467 & 13.042 & 5.194 & \textbf{3.354} & \textbf{200s} \\
\textbf{Total} & 1.125 & 358 & 135.986 & 144.674 & 130.424 & 51.935 & \textbf{33.539} & \textbf{200s} \\
\bottomrule
\end{tabular}
\end{table}

% ============================================
% TABELA 7: Impacto da Taxa de Destruição - GAP
% ============================================
\begin{table}[htbp]
\centering
\caption{Gap de otimalidade (\%) para diferentes taxas de destruição. Gap igual a zero indica solução ótima.}
\label{tab:lns_cand_gap}
\small
\begin{tabular}{@{}lrrrrrrrr@{}}
\toprule
\textbf{Instância} & \textbf{Disciplinas} & \textbf{Profs} & \textbf{Relax.} & \textbf{10\%} & \textbf{25\%} & \textbf{50\%} & \textbf{75\%} & \textbf{Melhor} \\
\midrule
input2  & 79  & 24 & 0,00 & 0,00 & 0,00 & 0,00 & 0,00 & Empate \\
input5  & 84  & 29 & 0,00 & 0,00 & 0,00 & 0,00 & 0,00 & Empate \\
input6  & 113 & 31 & 0,00 & 0,00 & 0,00 & 0,00 & 0,00 & Empate \\
input7  & 99  & 28 & 0,60 & 0,60 & 0,60 & 0,60 & 0,60 & Empate \\
input9  & 133 & 41 & 0,00 & 0,00 & 0,00 & 0,00 & 0,00 & Empate \\
input13 & 125 & 39 & 0,00 & 0,00 & 0,00 & 0,00 & 0,00 & Empate \\
input19 & 114 & 39 & 0,00 & 0,00 & 0,00 & 0,00 & 0,00 & Empate \\
input20 & 113 & 40 & 0,00 & 0,00 & 0,00 & 0,00 & 0,00 & Empate \\
input26 & 185 & 60 & 0,00 & 0,00 & 0,00 & 0,00 & 0,00 & Empate \\
input34 & 137 & 42 & 0,00 & 0,00 & 0,00 & 0,00 & 0,00 & Empate \\
\midrule
\textbf{Média}  & 114 & 39 & 0,06 & 0,06 & 0,06 & 0,06 & 0,06 & Empate \\
\textbf{Total}  & 1.296 & 412 & 0,60 & 0,60 & 0,60 & 0,60 & 0,60 & \textbf{Empate} \\
\bottomrule
\end{tabular}
\end{table}

% ============================================
% TABELA 8: Impacto da Ordenação - GAP
% ============================================
\begin{table}[htbp]
\centering
\caption{Gap de otimalidade (\%) comparando ordenação crescente e decrescente.}
\label{tab:lns_order_gap}
\small
\begin{tabular}{@{}lrrrrrr@{}}
\toprule
\textbf{Instância} & \textbf{Disciplinas} & \textbf{Profs} & \textbf{Relax.} & \textbf{Crescente} & \textbf{Decrescente} & \textbf{Melhor} \\
\midrule
input4  & 123 & 46 & 0,20 & 0,20 & 0,20 & Empate \\
input5  & 214 & 69 & 0,00 & 0,00 & 0,00 & Empate \\
input12 & 115 & 37 & 0,00 & 0,00 & 0,00 & Empate \\
input15 & 94  & 30 & 0,73 & 0,73 & 0,73 & Empate \\
input16 & 89  & 28 & 0,00 & 0,00 & 0,00 & Empate \\
input17 & 108 & 38 & 0,20 & 0,20 & 0,20 & Empate \\
input20 & 104 & 37 & 0,00 & 0,00 & 0,00 & Empate \\
input33 & 104 & 36 & 0,49 & 0,49 & 0,49 & Empate \\
input37 & 152 & 53 & 0,00 & 0,00 & 0,00 & Empate \\
input38 & 62  & 22 & 0,00 & 0,00 & 0,00 & Empate \\
\midrule
\textbf{Média}  & 106 & 37 & 0,16 & 0,16 & 0,16 & Empate \\
\textbf{Total}  & 1.271 & 433 & 1,62 & 1,62 & 1,62 & \textbf{Empate} \\
\bottomrule
\end{tabular}
\end{table}

% ============================================
% TABELA 9: Impacto do Tempo Máximo LNS - GAP
% ============================================
\begin{table}[htbp]
\centering
\caption{Gap de otimalidade (\%) para diferentes limites de tempo do sub-MIP.}
\label{tab:lns_time_gap}
\small
\begin{tabular}{@{}lrrrrrrrrr@{}}
\toprule
\textbf{Inst.} & \textbf{Disc.} & \textbf{Prof.} & \textbf{Relax.} & \textbf{10s} & \textbf{50s} & \textbf{100s} & \textbf{200s} & \textbf{Melhor} \\
\midrule
input6  & 79  & 24 & 0,00 & 0,00 & 0,00 & 0,00 & 0,00 & Empate \\
input14 & 99  & 28 & 0,00 & 0,00 & 0,00 & 0,00 & 0,00 & Empate \\
input15 & 94  & 30 & 0,73 & 0,73 & 0,73 & \textbf{0,00} & \textbf{0,00} & \textbf{100s} \\
input21 & 104 & 31 & 0,00 & 0,00 & 0,00 & 0,00 & 0,00 & Empate \\
input22 & 133 & 41 & 0,32 & 0,32 & 0,32 & 0,48 & 0,48 & Relax. \\
input23 & 118 & 35 & 0,00 & 0,00 & 0,00 & 0,00 & 0,00 & Empate \\
input27 & 125 & 39 & 0,00 & 0,00 & 0,00 & 0,00 & 0,00 & Empate \\
input29 & 103 & 39 & 0,19 & 0,19 & \textbf{0,00} & \textbf{0,00} & \textbf{0,00} & \textbf{50s} \\
input33 & 104 & 36 & 0,49 & 0,49 & 0,49 & \textbf{0,24} & \textbf{0,24} & \textbf{100s} \\
input38 & 62  & 22 & 0,00 & 0,00 & 0,00 & 0,00 & 0,00 & Empate \\
\midrule
\textbf{Média} & 104 & 33 & 0,17 & 0,17 & 0,15 & 0,07 & \textbf{0,07} & \textbf{100s} \\
\textbf{Total} & 1.125 & 358 & 1,73 & 1,73 & 1,54 & \textbf{0,72} & \textbf{0,72} & \textbf{100s} \\
\bottomrule
\end{tabular}
\end{table}

% ============================================
% TABELA 10: Avaliação Geral - Tempo
% ============================================
\begin{table}[htbp]
\centering
\caption{Tempo de execução (segundos) comparando Relaxação, LNS, GRASP e GRASP|LNS. Configuração LNS: 75\% destruição, crescente, 200s. Configuração GRASP: max\_iter=1, alpha=0.8, local\_search=0.}
\label{tab:geral_tempo}
\small
\begin{tabular}{@{}lrrrrrr@{}}
\toprule
\textbf{Instância} & \textbf{Disciplinas} & \textbf{Profs} & \textbf{Relax.} & \textbf{LNS} & \textbf{GRASP} & \textbf{GRASP|LNS} \\
\midrule
input5  & 214 & 69 & 112,63 & \textbf{54,46}  & 400,12 & 83,09 \\
input11 & 100 & 31 & 1,54   & 2,59   & 3,90   & \textbf{0,83} \\
input15 & 94  & 30 & 400,03 & \textbf{79,90}  & 400,02 & 400,02 \\
input16 & 89  & 28 & 3,07   & 0,75   & 8,38   & \textbf{0,36} \\
input19 & 95  & 29 & 2,03   & \textbf{1,45}   & 5,74   & 2,73 \\
input21 & 104 & 31 & 5,93   & \textbf{3,16}   & 13,45  & 4,42 \\
input25 & 107 & 34 & 15,63  & \textbf{9,27}   & 37,64  & 9,54 \\
input30 & 109 & 36 & 3,21   & \textbf{1,36}   & 7,94   & 2,36 \\
input35 & 169 & 53 & 186,72 & \textbf{63,27}  & 400,07 & 71,20 \\
input42 & 142 & 35 & \textbf{6,91}   & 8,18   & 20,19  & 13,39 \\
\midrule
\textbf{Média} & 106 & 33 & 73,76 & \textbf{22,44} & 129,75 & 58,79 \\
\textbf{Total} & 1.329 & 409 & 744,13 & \textbf{230,05} & 1.314,27 & 594,92 \\
\bottomrule
\end{tabular}
\end{table}

% ============================================
% TABELA 11: Avaliação Geral - Nodes Left
% ============================================
\begin{table}[htbp]
\centering
\caption{Número de nós restantes comparando Relaxação, LNS, GRASP e GRASP|LNS.}
\label{tab:geral_nodes}
\small
\begin{tabular}{@{}lrrrrrr@{}}
\toprule
\textbf{Instância} & \textbf{Disciplinas} & \textbf{Profs} & \textbf{Relax.} & \textbf{LNS} & \textbf{GRASP} & \textbf{GRASP|LNS} \\
\midrule
input5  & 214 & 69 & 0      & \textbf{0}      & 8.037  & \textbf{0} \\
input11 & 100 & 31 & 0      & \textbf{0}      & 0      & \textbf{0} \\
input15 & 94  & 30 & 64.364 & \textbf{0}      & 7.526  & 2.324 \\
input16 & 89  & 28 & 0      & \textbf{0}      & 0      & \textbf{0} \\
input19 & 95  & 29 & 0      & \textbf{0}      & 0      & \textbf{0} \\
input21 & 104 & 31 & 0      & \textbf{0}      & 0      & \textbf{0} \\
input25 & 107 & 34 & 0      & \textbf{0}      & 0      & \textbf{0} \\
input30 & 109 & 36 & 0      & \textbf{0}      & 0      & \textbf{0} \\
input35 & 169 & 53 & 0      & \textbf{0}      & 20.740 & \textbf{0} \\
input42 & 142 & 35 & 0      & \textbf{0}      & 0      & \textbf{0} \\
\midrule
\textbf{Média} & 106 & 33 & 6.436 & \textbf{0} & 3.630 & 232 \\
\textbf{Total} & 1.329 & 409 & 64.364 & \textbf{0} & 36.303 & 2.324 \\
\bottomrule
\end{tabular}
\end{table}

% ============================================
% TABELA 12: Avaliação Geral - GAP
% ============================================
\begin{table}[htbp]
\centering
\caption{Gap de otimalidade (\%) comparando Relaxação, LNS, GRASP e GRASP|LNS.}
\label{tab:geral_gap}
\small
\begin{tabular}{@{}lrrrrrr@{}}
\toprule
\textbf{Instância} & \textbf{Disciplinas} & \textbf{Profs} & \textbf{Relax.} & \textbf{LNS} & \textbf{GRASP} & \textbf{GRASP|LNS} \\
\midrule
input5  & 214 & 69 & 0,00 & \textbf{0,00} & 1,01 & \textbf{0,00} \\
input11 & 100 & 31 & 0,00 & \textbf{0,00} & 0,00 & \textbf{0,00} \\
input15 & 94  & 30 & 0,73 & \textbf{0,00} & 0,24 & 0,24 \\
input16 & 89  & 28 & 0,00 & \textbf{0,00} & 0,00 & \textbf{0,00} \\
input19 & 95  & 29 & 0,00 & \textbf{0,00} & 0,00 & \textbf{0,00} \\
input21 & 104 & 31 & 0,00 & \textbf{0,00} & 0,00 & \textbf{0,00} \\
input25 & 107 & 34 & 0,00 & \textbf{0,00} & 0,00 & \textbf{0,00} \\
input30 & 109 & 36 & 0,00 & \textbf{0,00} & 0,00 & \textbf{0,00} \\
input35 & 169 & 53 & 0,00 & \textbf{0,00} & 2,21 & \textbf{0,00} \\
input42 & 142 & 35 & 0,00 & \textbf{0,00} & 0,00 & \textbf{0,00} \\
\midrule
\textbf{Média} & 106 & 33 & 0,07 & \textbf{0,00} & 0,35 & 0,02 \\
\textbf{Total} & 1.329 & 409 & 0,73 & \textbf{0,00} & 3,46 & 0,24 \\
\bottomrule
\end{tabular}
\end{table}


\textbf{Tempo de Execução.} A Tabela~\ref{tab:lns_cand_tempo} apresenta os tempos de execução para as 10 instâncias selecionadas. A configuração de relaxação (sem LNS) obteve o menor tempo médio, com 19,49 segundos, contra 25,39 segundos da melhor configuração LNS (75\%). Esse resultado é esperado, pois o \emph{overhead} introduzido pelas chamadas ao sub-MIP do LNS aumenta o tempo total, sobretudo em instâncias que já são resolvidas de forma ótima pelo Branch-and-Bound puro. Nove das dez instâncias foram resolvidas até a otimalidade por todas as configurações, e apenas a instância input7 atingiu o limite de tempo. Assim, o tempo de execução não foi suficiente para diferenciar as configurações LNS entre si.

\textbf{Gap de Otimalidade.} Em termos de gap, todas as configurações --- incluindo a relaxação --- apresentaram desempenho equivalente (0,60\% de gap total), com apenas a instância input7 permanecendo sem solução ótima comprovada (ver Tabela~\ref{tab:lns_cand_gap}). Esse empate impediu o uso do gap como critério de seleção.

\textbf{Nós Restantes (\emph{Nodes Left}).} Uma vez que tanto o tempo quanto o gap resultaram em empate, o critério de desempate utilizado foi a quantidade de nós restantes na árvore de Branch-and-Bound ao final da execução, métrica que reflete o progresso em direção à prova de otimalidade. A Tabela~\ref{tab:lns_cand_nodes} revela que a configuração com 75\% de destruição alcançou o menor total de nós restantes: \textbf{58.921 nós}, comparado a 66.230 (relaxação), 62.628 (10\%), 60.540 (25\%) e 61.056 (50\%). Todas as configurações obtiveram resultado idêntico em 9 das 10 instâncias (resolvidas até a otimalidade), mas na instância crítica input7, a taxa de 75\% demonstrou maior capacidade de exploração e poda da árvore. A Figura~\ref{fig:nodes_cand} apresenta visualmente a comparação dos totais, e a Figura~\ref{fig:avg_nodes_cand} exibe a média por instância, evidenciando a superioridade da configuração de 75\%.

% Gráfico: Taxa de Destruição - Nodes Left
\begin{figure}[htbp]
\centering
\begin{tikzpicture}
\begin{axis}[
    ybar,
    bar width=20pt,
    width=0.9\textwidth,
    height=8cm,
    ylabel={Nós Restantes (Total)},
    xlabel={Taxa de Destruição},
    symbolic x coords={Relaxação, 10\%, 25\%, 50\%, 75\%},
    xtick=data,
    nodes near coords,
    nodes near coords align={vertical},
    nodes near coords style={font=\footnotesize},
    ymin=0,
    ymax=70000,
    enlarge x limits=0.15,
    ylabel style={font=\large},
    xlabel style={font=\large},
    tick label style={font=\normalsize},
]
\addplot[fill=blue!60] coordinates {
    (Relaxação,66230)
    (10\%,62628)
    (25\%,60540)
    (50\%,61056)
    (75\%,58921)
};
\end{axis}
\end{tikzpicture}
\caption{Número total de nós restantes na árvore de Branch-and-Bound para diferentes taxas de destruição do LNS. A configuração de 75\% apresentou o melhor desempenho com 58.921 nós restantes.}
\label{fig:nodes_cand}
\end{figure}


% Gráfico: Taxa de Destruição - Média de Nós Restantes
\begin{figure}[htbp]
\centering
\begin{tikzpicture}
\begin{axis}[
    ybar,
    bar width=20pt,
    width=0.9\textwidth,
    height=8cm,
    ylabel={Média de Nós Restantes},
    xlabel={Taxa de Destruição},
    symbolic x coords={Relaxação, 10\%, 25\%, 50\%, 75\%},
    xtick=data,
    nodes near coords,
    nodes near coords align={vertical},
    nodes near coords style={font=\footnotesize},
    every node near coord/.append style={/pgf/number format/fixed,
        /pgf/number format/precision=0,
        /pgf/number format/use comma},
    ymin=0,
    ymax=8000,
    enlarge x limits=0.15,
    ylabel style={font=\large},
    xlabel style={font=\large},
    tick label style={font=\normalsize},
]
\addplot[fill=blue!60] coordinates {
    (Relaxação,6623)
    (10\%,6263)
    (25\%,6054)
    (50\%,6106)
    (75\%,5892)
};
\end{axis}
\end{tikzpicture}
\caption{Média de nós restantes na árvore de Branch-and-Bound para diferentes taxas de destruição do LNS. A configuração de 75\% apresentou a menor média, com 5.892 nós restantes por instância.}
\label{fig:avg_nodes_cand}
\end{figure}


\textbf{Conclusão Parcial.} A taxa de destruição de \textbf{75\%} foi selecionada para as análises subsequentes, pois demonstrou melhor capacidade de redução da árvore de busca na instância difícil. Esse resultado indica que vizinhanças maiores permitem ao LNS realizar explorações mais agressivas e produtivas do espaço de soluções, reforçando a estratégia de intensificação adotada pela matheurística.

\subsection{Impacto do Sentido de Ordenação (Crescente vs. Decrescente)}

Com a taxa de destruição fixada em 75\%, investigou-se o impacto do sentido de ordenação dos candidatos pelo valor de \texttt{avgPreferenceWeight}. Conforme descrito no Capítulo~\ref{ch:dpd}, a ordenação crescente prioriza a liberação de professores mais versáteis (maior número de preferências declaradas), enquanto a decrescente prioriza professores mais especializados (menor número de preferências).

\textbf{Tempo de Execução.} A Tabela~\ref{tab:lns_order_tempo} revela desempenho temporal praticamente idêntico entre as duas estratégias: 180,31 segundos (crescente) versus 180,38 segundos (decrescente) no tempo médio, com diferença inferior a 0,1 segundo. Das 10 instâncias testadas, quatro atingiram o limite de tempo em ambas as configurações (input4, input15, input17 e input33), e as demais apresentaram variações inferiores a 1 segundo. O tempo, portanto, não permitiu diferenciar as estratégias.

\textbf{Gap de Otimalidade.} Ambas as configurações mantiveram gap total idêntico de 1,62\%, distribuído entre quatro instâncias não resolvidas de forma ótima (input4, input15, input17 e input33), com gaps individuais variando de 0,20\% a 0,73\% (ver Tabela~\ref{tab:lns_order_gap}). O empate no gap impediu, novamente, seu uso como critério de seleção.

\textbf{Nós Restantes (\emph{Nodes Left}).} A métrica de nós restantes revelou vantagem para a ordenação crescente. A Tabela~\ref{tab:lns_order_nodes} mostra que a configuração crescente resultou em \textbf{131.997 nós restantes} no total, enquanto a decrescente deixou 133.674 nós --- uma redução de 1.677 nós (1,25\%). A diferença se concentra em duas instâncias críticas: input15 (61.546 vs.\ 62.452 nós) e input33 (46.243 vs.\ 47.063 nós). A Figura~\ref{fig:nodes_order} apresenta a comparação visual dos totais, e a Figura~\ref{fig:avg_nodes_order} exibe a média por instância.

% Gráfico: Ordenação - Nodes Left
\begin{figure}[htbp]
\centering
\begin{tikzpicture}
\begin{axis}[
    ybar,
    bar width=25pt,
    width=0.85\textwidth,
    height=8cm,
    ylabel={Nós Restantes (Total)},
    xlabel={Sentido de Ordenação},
    symbolic x coords={Relaxação, Crescente, Decrescente},
    xtick=data,
    nodes near coords,
    nodes near coords align={vertical},
    nodes near coords style={font=\normalsize},
    ymin=0,
    ymax=150000,
    enlarge x limits=0.25,
    ylabel style={font=\large},
    xlabel style={font=\large},
    tick label style={font=\normalsize},
]
\addplot[fill=blue!70] coordinates {
    (Relaxação,139460)
    (Crescente,131997)
    (Decrescente,133674)
};
\end{axis}
\end{tikzpicture}
\caption{Comparação do número total de nós restantes entre ordenação crescente e decrescente. A ordenação crescente (priorizando professores versáteis) obteve melhor desempenho com 131.997 nós restantes.}
\label{fig:nodes_order}
\end{figure}


% Gráfico: Ordenação - Média de Nós Restantes
\begin{figure}[htbp]
\centering
\begin{tikzpicture}
\begin{axis}[
    ybar,
    bar width=25pt,
    width=0.85\textwidth,
    height=8cm,
    ylabel={Média de Nós Restantes},
    xlabel={Sentido de Ordenação},
    symbolic x coords={Relaxação, Crescente, Decrescente},
    xtick=data,
    nodes near coords,
    nodes near coords align={vertical},
    nodes near coords style={font=\normalsize},
    every node near coord/.append style={/pgf/number format/fixed,
        /pgf/number format/precision=0,
        /pgf/number format/use comma},
    ymin=0,
    ymax=18000,
    enlarge x limits=0.25,
    ylabel style={font=\large},
    xlabel style={font=\large},
    tick label style={font=\normalsize},
]
\addplot[fill=blue!70] coordinates {
    (Relaxação,13946)
    (Crescente,13200)
    (Decrescente,13367)
};
\end{axis}
\end{tikzpicture}
\caption{Média de nós restantes por instância comparando ordenação crescente e decrescente. A ordenação crescente obteve a menor média, com 13.200 nós restantes.}
\label{fig:avg_nodes_order}
\end{figure}


\textbf{Interpretação.} A superioridade da ordenação \textbf{crescente} sugere que, para o DPD, priorizar a realocação de professores mais versáteis (generalistas) oferece maior flexibilidade para o sub-MIP encontrar rearranjos que melhorem a solução global. Professores generalistas possuem preferências por um maior número de disciplinas, o que amplia o espaço de busca do sub-MIP e facilita a satisfação simultânea de múltiplas restrições de carga horária. A ordenação crescente foi, portanto, adotada para as análises subsequentes.

\subsection{Impacto do Tempo Máximo do LNS (10s, 50s, 100s e 200s)}

Com os parâmetros de taxa de destruição (75\%) e ordenação (crescente) definidos, investigou-se o impacto do limite de tempo do sub-MIP (\texttt{lns\_time}). Foram testados quatro valores: 10s, 50s, 100s e 200s, mantendo o limite global de 400 segundos por instância.

\textbf{Tempo de Execução.} A Tabela~\ref{tab:lns_time_tempo} apresenta resultados contraintuitivos: embora se esperasse que limites maiores de LNS aumentassem o tempo total, observou-se comportamento não monotônico. O tempo médio foi de 164,77s (10s), 162,11s (50s), 160,27s (100s) e \textbf{159,46s (200s)}. O tempo total agregado reforça essa tendência: 1.629,16s (relaxação), 1.647,01s (10s), 1.313,60s (50s), \textbf{1.039,68s (100s)} e 1.090,00s (200s). Destaca-se a instância input15, em que as configurações de 100s e 200s resolveram o problema em $\sim$77s, contra 400s nas demais. Contudo, os resultados de tempo médio e tempo total apontam para configurações diferentes (200s e 100s, respectivamente), não permitindo uma conclusão unívoca por este critério.

Essa redução de tempo com limites maiores de LNS pode ser explicada pela capacidade de encontrar soluções de melhor qualidade mais cedo: uma incumbente mais forte permite podas mais agressivas na árvore de Branch-and-Bound, acelerando a convergência. Configurações com tempo insuficiente (10s) produzem melhorias marginais que não compensam o \emph{overhead} da heurística.

\textbf{Gap de Otimalidade.} A Tabela~\ref{tab:lns_time_gap} mostra que as configurações de 100s e 200s empataram com gap total de \textbf{0,72\%}, substancialmente inferior ao obtido pela relaxação e pelo LNS com 10s (ambos com 1,73\%). O destaque vai para a instância input15, resolvida até a otimalidade (gap 0,00\%) apenas pelas configurações de 100s e 200s, e para a instância input33, cujo gap caiu de 0,49\% (relaxação) para 0,24\% com 100s e 200s. Entretanto, o empate entre 100s e 200s impediu o uso do gap como critério final.

\textbf{Nós Restantes (\emph{Nodes Left}).} A Tabela~\ref{tab:lns_time_nodes} foi o critério de desempate definitivo. O total de nós restantes foi: 135.986 (relaxação), 144.674 (10s), 130.424 (50s), 51.935 (100s) e \textbf{33.539 (200s)}. A configuração de 200s reduziu os nós em 75,3\% em relação à relaxação e em 76,8\% em relação ao LNS com 10s. As instâncias input15 e input29 foram resolvidas completamente (0 nós restantes) pelas configurações de 100s e 200s, enquanto permaneceram com dezenas de milhares de nós nas configurações inferiores. Na instância input33, a configuração de 200s deixou apenas 539 nós, contra 1.524 com 100s. A Figura~\ref{fig:nodes_time} apresenta os totais, e a Figura~\ref{fig:avg_nodes_time} ilustra a média por instância, evidenciando a clara superioridade do limite de 200s.

% Gráfico: Tempo Máximo LNS - Nodes Left
\begin{figure}[htbp]
\centering
\begin{tikzpicture}
\begin{axis}[
    ybar,
    bar width=20pt,
    width=0.9\textwidth,
    height=8cm,
    ylabel={Nós Restantes (Total)},
    xlabel={Tempo Máximo do Sub-MIP (segundos)},
    symbolic x coords={Relaxação, 10s, 50s, 100s, 200s},
    xtick=data,
    nodes near coords,
    nodes near coords align={vertical},
    nodes near coords style={font=\footnotesize},
    ymin=0,
    ymax=160000,
    enlarge x limits=0.15,
    ylabel style={font=\large},
    xlabel style={font=\large},
    tick label style={font=\normalsize},
]
\addplot[fill=blue!60] coordinates {
    (Relaxação,135986)
    (10s,144674)
    (50s,130424)
    (100s,51935)
    (200s,33539)
};
\end{axis}
\end{tikzpicture}
\caption{Impacto do limite de tempo do sub-MIP no número de nós restantes. A configuração de 200s demonstrou o melhor desempenho com 33.539 nós, representando uma redução de 75,3\% em relação à relaxação pura.}
\label{fig:nodes_time}
\end{figure}


% Gráfico: Tempo Máximo LNS - Média de Nós Restantes
\begin{figure}[htbp]
\centering
\begin{tikzpicture}
\begin{axis}[
    ybar,
    bar width=20pt,
    width=0.9\textwidth,
    height=8cm,
    ylabel={Média de Nós Restantes},
    xlabel={Tempo Máximo do Sub-MIP (segundos)},
    symbolic x coords={Relaxação, 10s, 50s, 100s, 200s},
    xtick=data,
    nodes near coords,
    nodes near coords align={vertical},
    nodes near coords style={font=\footnotesize},
    every node near coord/.append style={/pgf/number format/fixed,
        /pgf/number format/precision=0,
        /pgf/number format/use comma},
    ymin=0,
    ymax=18000,
    enlarge x limits=0.15,
    ylabel style={font=\large},
    xlabel style={font=\large},
    tick label style={font=\normalsize},
]
\addplot[fill=blue!60] coordinates {
    (Relaxação,13599)
    (10s,14467)
    (50s,13042)
    (100s,5194)
    (200s,3354)
};
\end{axis}
\end{tikzpicture}
\caption{Média de nós restantes por instância para diferentes limites de tempo do sub-MIP. A configuração de 200s alcançou a menor média, com 3.354 nós, representando uma redução de 75,3\% em relação à relaxação pura.}
\label{fig:avg_nodes_time}
\end{figure}


\textbf{Conclusão Parcial.} O limite de \textbf{200 segundos} demonstrou o melhor desempenho global, reduzindo drasticamente a árvore de busca e resolvendo instâncias que outras configurações não conseguiram. Para problemas combinatórios como o DPD, investir mais tempo em cada chamada do LNS é compensado por soluções de maior qualidade, que aceleram a convergência do Branch-and-Bound. O \emph{trade-off} entre intensidade (tempo por chamada) e frequência (número de chamadas) favorece claramente a intensidade para este problema.

\subsection{Resumo da Configuração Selecionada}

A Tabela~\ref{tab:config_final} resume os parâmetros selecionados ao longo da análise incremental.

\begin{table}[htbp]
\centering
\caption{Configuração final da heurística LNS selecionada pela análise incremental.}
\label{tab:config_final}
\small
\begin{tabular}{@{}llll@{}}
\toprule
\textbf{Parâmetro} & \textbf{Valores Testados} & \textbf{Selecionado} & \textbf{Critério Decisivo} \\
\midrule
Taxa de destruição (\texttt{lns\_perc}) & 10\%, 25\%, 50\%, 75\% & 75\% & Nós restantes \\
Ordenação (\texttt{lns\_order})         & Crescente, Decrescente  & Crescente & Nós restantes \\
Tempo do sub-MIP (\texttt{lns\_time})   & 10s, 50s, 100s, 200s   & 200s & Nós restantes \\
\bottomrule
\end{tabular}
\end{table}

Nota-se que, em todas as etapas, o tempo de execução e o gap de otimalidade não foram suficientes para diferenciar as configurações, sendo a métrica de nós restantes o critério de desempate definitivo. Esse padrão sugere que, para as instâncias avaliadas, o LNS atua predominantemente como acelerador da prova de otimalidade, reduzindo o esforço do Branch-and-Bound por meio de incumbentes de melhor qualidade.

\section{Comparativo de Desempenho: LNS, GRASP e Modelo Exato}

Com os parâmetros da LNS calibrados (75\% de destruição, ordenação crescente e tempo de sub-MIP de 200s), realizou-se um comparativo final entre quatro configurações: (i)~\textbf{Relaxação} (Branch-and-Bound puro, sem heurísticas primais customizadas), (ii)~\textbf{LNS} (apenas a heurística LNS proposta), (iii)~\textbf{GRASP} (apenas a heurística GRASP, com \texttt{max\_iter=1}, $\alpha=0{,}8$ e sem busca local \cite{jhonatan_grasp_dpd}) e (iv)~\textbf{GRASP|LNS} (ambas as heurísticas ativadas simultaneamente). As mesmas 10 instâncias foram utilizadas em todas as configurações.

\subsection{Análise do Tempo de Execução}

A Tabela~\ref{tab:geral_tempo} apresenta os tempos de execução para cada configuração. O \textbf{LNS} obteve o menor tempo médio: \textbf{22,44 segundos}, representando uma redução de 69,6\% em relação à relaxação (73,76s) e de 82,7\% em relação ao GRASP (129,75s). A configuração GRASP|LNS alcançou 58,79s de tempo médio, posicionando-se entre a relaxação e o LNS.

Dentre as 10 instâncias, o LNS foi o mais rápido em 7 delas, com destaque para a instância input15: enquanto a relaxação e o GRASP atingiram o limite de 400s, o LNS resolveu a instância em 79,90s. A instância input35, com 169 disciplinas e 53 professores, exemplifica o impacto da heurística: o LNS concluiu em 63,27s contra 186,72s da relaxação e 400,07s do GRASP. A Figura~\ref{fig:geral_tempo} ilustra a comparação dos tempos médios.

% Gráfico: Comparativo Geral - Tempo Médio
\begin{figure}[htbp]
\centering
\begin{tikzpicture}
\begin{axis}[
    ybar,
    bar width=22pt,
    width=0.9\textwidth,
    height=8cm,
    ylabel={Tempo Médio de Execução (segundos)},
    xlabel={Configuração},
    symbolic x coords={Relaxação, LNS, GRASP, GRASP{|}LNS},
    xtick=data,
    nodes near coords,
    nodes near coords align={vertical},
    nodes near coords style={font=\normalsize, /pgf/number format/fixed,
        /pgf/number format/precision=2},
    ymin=0,
    ymax=160,
    enlarge x limits=0.2,
    ylabel style={font=\large},
    xlabel style={font=\large},
    tick label style={font=\normalsize},
]
\addplot[fill=blue!50] coordinates {
    (Relaxação,73.76)
    (LNS,22.44)
    (GRASP,129.75)
    (GRASP{|}LNS,58.79)
};
\end{axis}
\end{tikzpicture}
\caption{Tempo médio de execução (segundos) para cada configuração. O LNS obteve o menor tempo médio (22,44s), representando uma redução de 69,6\% em relação à relaxação pura e de 82,7\% em relação ao GRASP.}
\label{fig:geral_tempo}
\end{figure}


O GRASP apresentou o pior desempenho temporal, atingindo o limite de tempo em 3 das 10 instâncias (input5, input15 e input35). Esse resultado sugere que a heurística construtiva do GRASP, embora forneça soluções iniciais viáveis rapidamente, não contribui de forma eficaz para a redução da árvore de busca do Branch-and-Bound.

\subsection{Análise do Gap de Otimalidade}

A Tabela~\ref{tab:geral_gap} apresenta o gap de otimalidade para cada configuração. O \textbf{LNS} atingiu \textbf{gap total de 0,00\%}, provando a otimalidade de todas as 10 soluções encontradas. A relaxação obteve 0,73\% (devido à instância input15 não resolvida), a GRASP|LNS obteve 0,24\% (input15 com gap residual) e o GRASP apresentou o pior resultado com 3,46\% de gap total, com três instâncias sem solução ótima comprovada (input5 com 1,01\%, input15 com 0,24\% e input35 com 2,21\%). A Figura~\ref{fig:geral_gap} apresenta a comparação visual.

% Gráfico: Comparativo Geral - Gap de Otimalidade
\begin{figure}[htbp]
\centering
\begin{tikzpicture}
\begin{axis}[
    ybar,
    bar width=22pt,
    width=0.9\textwidth,
    height=8cm,
    ylabel={Gap de Otimalidade Total (\%)},
    xlabel={Configuração},
    symbolic x coords={Relaxação, LNS, GRASP, GRASP{|}LNS},
    xtick=data,
    nodes near coords,
    nodes near coords align={vertical},
    nodes near coords style={font=\normalsize, /pgf/number format/fixed,
        /pgf/number format/precision=2},
    ymin=0,
    ymax=4.5,
    enlarge x limits=0.2,
    ylabel style={font=\large},
    xlabel style={font=\large},
    tick label style={font=\normalsize},
]
\addplot[fill=blue!60] coordinates {
    (Relaxação,0.73)
    (LNS,0.00)
    (GRASP,3.46)
    (GRASP{|}LNS,0.24)
};
\end{axis}
\end{tikzpicture}
\caption{Gap de otimalidade total (\%) para cada configuração. O LNS atingiu gap zero, provando a otimalidade de todas as soluções encontradas.}
\label{fig:geral_gap}
\end{figure}


A superioridade do LNS no gap é particularmente expressiva: enquanto a relaxação pura não conseguiu provar a otimalidade da instância input15 em 400s, o LNS resolveu-a completamente. O GRASP, por sua vez, não apenas falhou em provar otimalidade, como também obteve soluções de pior qualidade em duas instâncias (input5 e input35), conforme indicado pelos gaps de 1,01\% e 2,21\%.

\subsection{Análise dos Nós Restantes}

A Tabela~\ref{tab:geral_nodes} confirma a superioridade do LNS. O \textbf{LNS alcançou 0 nós restantes} no total, indicando que todas as instâncias foram resolvidas até a otimalidade dentro do limite de tempo. A relaxação deixou 64.364 nós (todos da instância input15), o GRASP deixou 36.303 nós (distribuídos entre input5, input15 e input35) e a GRASP|LNS deixou 2.324 nós (apenas input15). A Figura~\ref{fig:geral_nodes} apresenta a comparação.

% Gráfico: Comparativo Geral - Nós Restantes (Total)
\begin{figure}[htbp]
\centering
\begin{tikzpicture}
\begin{axis}[
    ybar,
    bar width=22pt,
    width=0.9\textwidth,
    height=8cm,
    ylabel={Nós Restantes (Total)},
    xlabel={Configuração},
    symbolic x coords={Relaxação, LNS, GRASP, GRASP{|}LNS},
    xtick=data,
    nodes near coords,
    nodes near coords align={vertical},
    nodes near coords style={font=\normalsize},
    every node near coord/.append style={/pgf/number format/fixed,
        /pgf/number format/precision=0,
        /pgf/number format/use comma},
    ymin=0,
    ymax=75000,
    enlarge x limits=0.2,
    ylabel style={font=\large},
    xlabel style={font=\large},
    tick label style={font=\normalsize},
]
\addplot[fill=blue!60] coordinates {
    (Relaxação,64364)
    (LNS,0)
    (GRASP,36303)
    (GRASP{|}LNS,2324)
};
\end{axis}
\end{tikzpicture}
\caption{Total de nós restantes na árvore de Branch-and-Bound para cada configuração. O LNS alcançou 0 nós restantes, resolvendo todas as instâncias até a otimalidade.}
\label{fig:geral_nodes}
\end{figure}


\subsection{Discussão dos Resultados}

Os resultados do comparativo evidenciam que a matheurística LNS proposta supera consistentemente tanto o Branch-and-Bound puro quanto o GRASP em todas as métricas avaliadas. A Tabela~\ref{tab:resumo_comparativo} sintetiza os principais indicadores.

\begin{table}[htbp]
\centering
\caption{Resumo comparativo das quatro configurações avaliadas.}
\label{tab:resumo_comparativo}
\small
\begin{tabular}{@{}lrrrr@{}}
\toprule
\textbf{Métrica} & \textbf{Relaxação} & \textbf{LNS} & \textbf{GRASP} & \textbf{GRASP|LNS} \\
\midrule
Tempo médio (s)        & 73,76  & \textbf{22,44} & 129,75 & 58,79 \\
Tempo total (s)        & 744,13 & \textbf{230,05} & 1.314,27 & 594,92 \\
Gap total (\%)         & 0,73   & \textbf{0,00}  & 3,46   & 0,24 \\
Nós restantes (total)  & 64.364 & \textbf{0}     & 36.303 & 2.324 \\
Instâncias ótimas      & 9/10   & \textbf{10/10} & 7/10   & 9/10 \\
\bottomrule
\end{tabular}
\end{table}

A análise revela três observações principais:

\begin{enumerate}
  \item \textbf{O LNS é o método mais eficaz.} Com a configuração calibrada (75\% de destruição, ordenação crescente, 200s de sub-MIP), o LNS resolveu todas as instâncias até a otimalidade, com tempo médio 3,3 vezes menor que a relaxação e 5,8 vezes menor que o GRASP. A estratégia de intensificação via sub-MIP gera incumbentes de alta qualidade que viabilizam podas mais eficientes na árvore de busca.

  \item \textbf{A combinação GRASP|LNS não supera o LNS isolado.} Embora a GRASP|LNS tenha apresentado desempenho superior à relaxação e ao GRASP, o \emph{overhead} da heurística construtiva GRASP não trouxe benefícios adicionais quando o LNS já estava ativo. Isso sugere que o LNS é capaz de encontrar incumbentes de qualidade suficiente sem a necessidade de uma solução inicial heurística dedicada.

  \item \textbf{O GRASP isolado apresenta limitações para o DPD.} O GRASP atingiu o limite de tempo em 30\% das instâncias e não conseguiu provar a otimalidade em 3 delas. A heurística construtiva, embora gere soluções viáveis rapidamente, não contribui de forma significativa para a redução do espaço de busca, e o \emph{overhead} computacional de sua execução a cada nó penaliza o desempenho global.
\end{enumerate}

% ----------------------------------------------------------
% Conclusão
% ----------------------------------------------------------
\chapter{Conclusão}

\section{Considerações Finais}

\section{Limitações do Estudo}

\section{Trabalhos Futuros}



% ----------------------------------------------------------
% Referências bibliográficas
% ----------------------------------------------------------
% \bibliographystyle{ieeetr}

% Bibliografia: aplicar \sloppy localmente para evitar overfull hboxes
\begingroup\sloppy
\bibliography{Auxiliares/References} %% REFERENCIA AO ARQUIVO abntex2-modelo-references.bib
\endgroup
\cleardoublepage %% Pula página

%\chapter{Referências}
% ---

% ----------------------------------------------------------
% ELEMENTOS PÓS-TEXTUAIS
% ----------------------------------------------------------
\postextual

% ----------------------------------------------------------
% Glossário
% ----------------------------------------------------------
%
% Consulte o manual da classe abntex2 para orientações sobre o glossário.
%
%\glossary

% ----------------------------------------------------------
% Apêndices
% ----------------------------------------------------------

% ---
% Inicia os apêndices
% ---
\begin{apendicesenv}

% Imprime uma página indicando o início dos apêndices
\partapendices

% ----------------------------------------------------------
\chapter{Códigos} \label{apendix}
% ----------------------------------------------------------

\lstset{language=Matlab,%
    %basicstyle=\color{red},
    breaklines=true,%
    morekeywords={matlab2tikz},
    keywordstyle=\color{blue},%
    morekeywords=[2]{1}, keywordstyle=[2]{\color{black}},
    identifierstyle=\color{black},%
    stringstyle=\color{mylilas},
    commentstyle=\color{mygreen},%
    showstringspaces=false,%without this there will be a symbol in the places where there is a space
    numbers=left,%
    numberstyle={\tiny \color{black}},% size of the numbers
    numbersep=9pt, % this defines how far the numbers are from the text
    emph=[1]{for,end,break},emphstyle=[1]\color{red}, %some words to emphasise
    %emph=[2]{word1,word2}, emphstyle=[2]{style},    
}

% Apêndices ou Anexos
\section{Apêndices - Materiais do autor} \label{sec:cod1}
Explicações

%exibição de código fonte
% \lstinputlisting{Codes/All_data.m}


\end{apendicesenv}



% ----------------------------------------------------------
% Anexos
% ----------------------------------------------------------

% ---
% Inicia os anexos
% ---
% \begin{anexosenv}

% Imprime uma página indicando o início dos anexos
% \partanexos

% ---
% \chapter{Morbi ultrices rutrum lorem.}
% ---

% \end{anexosenv}

\end{document}